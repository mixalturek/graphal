%%%%%%%%%%%%%%%%%%%%%%%%%%%%%%%%%%%%%%%%%%%%%%%%%%%%%%%%%%%%%%%%%%%%%%%%%%%%%
%%%%

%% History:
% Pavel Tvrdik (26.12.2004)
%  + initial version for PhD Report
%
% Daniel Sykora (27.01.2005)
%
% Michal Valenta (3.12.2008)
% rada zmen ve formatovani (diky M. Duškovi, J. Holubovi a J. Žďárkovi)
% sjednoceni zdrojoveho kodu pro anglickou, ceskou, bakalarskou a diplomovou praci


%%%%%%%%%%%%%%%%%%%%%%%%%%%%%%%%%%%%%%%%%%%%%%%%%%%%%%%%%%%%%%%%%%%%%%%%%%%%%
%%%%

\documentclass[11pt,twoside,a4paper]{book}
\usepackage[czech, english]{babel}

%\usepackage[T1]{fontenc} % pouzije EC fonty
\usepackage[utf8]{inputenc}
\usepackage{graphicx}
%\usepackage{indentfirst} %1. odstavec jako v cestine.

\usepackage{k336_thesis_macros}


% \newcommand\TypeOfWork{Diplomová práce} \typeout{Diplomova prace}
\newcommand\TypeOfWork{Master's Thesis}   \typeout{Master's Thesis}


% http://www.fel.cvut.cz/cz/education/bk/prehled.html

% \newcommand\StudProgram{Elektrotechnika a informatika, strukturovaný, Navazující magisterský}
\newcommand\StudProgram{Electrical Engineering and Information Technology}  % master program

%\newcommand\StudBranch{Výpočetní technika}   % pro program EaI bak. (dobihajici i strukt.)
%\newcommand\StudBranch{Výpočetní technika}   % pro prgoram EaI mag. (dobihajici i strukt.)
%\newcommand\StudBranch{Computer Engineering}             % bachelor programe
\newcommand\StudBranch{Computer Science and Engineering}  % master programe


\newcommand\WorkTitle{Graph algorithms interpreter}
\newcommand\FirstandFamilyName{Bc. Michal Turek}
\newcommand\Supervisor{RNDr. Marko Genyk-Berezovskyj}


%%%%%%%%%%%%%%%%%%%%%%%%%%%%%%%%%%%%%%%%%%%%%%%%%%%%%%%%%%%%%%%%%%%%%%%%%%%%%
%%%%

\usepackage[
pdftitle={\WorkTitle},
pdfauthor={\FirstandFamilyName},
bookmarks=true,
colorlinks=true,
breaklinks=true,
urlcolor=red,
citecolor=blue,
linkcolor=blue,
unicode=true,
]
{hyperref}


%%%%%%%%%%%%%%%%%%%%%%%%%%%%%%%%%%%%%%%%%%%%%%%%%%%%%%%%%%%%%%%%%%%%%%%%%%%%%
%%%%

\begin{document}

%\selectlanguage{czech}
\selectlanguage{english}

\iflanguage{czech}{
	 \typeout{************************************************}
	 \typeout{Zvoleny jazyk: cestina}
	 \typeout{Typ prace: \TypeOfWork}
	 \typeout{Studijni program: \StudProgram}
	 \typeout{Obor: \StudBranch}
	 \typeout{Jmeno: \FirstandFamilyName}
	 \typeout{Nazev prace: \WorkTitle}
	 \typeout{Vedouci prace: \Supervisor}
	 \typeout{***************************************************}
	 \newcommand\Department{Katedra počítačů}
	 \newcommand\Faculty{Fakulta elektrotechnická}
	 \newcommand\University{České vysoké učení technické v Praze}
	 \newcommand\labelSupervisor{Vedoucí práce}
	 \newcommand\labelStudProgram{Studijní program}
	 \newcommand\labelStudBranch{Obor}
}{
	 \typeout{************************************************}
	 \typeout{Language: english}
	 \typeout{Type of Work: \TypeOfWork}
	 \typeout{Study Program: \StudProgram}
	 \typeout{Study Branch: \StudBranch}
	 \typeout{Author: \FirstandFamilyName}
	 \typeout{Title: \WorkTitle}
	 \typeout{Supervisor: \Supervisor}
	 \typeout{***************************************************}
	 \newcommand\Department{Department of Computer Science and Engineering}
	 \newcommand\Faculty{Faculty of Electrical Engineering}
	 \newcommand\University{Czech Technical University in Prague}
	 \newcommand\labelSupervisor{Supervisor}
	 \newcommand\labelStudProgram{Study Programme}
	 \newcommand\labelStudBranch{Field of Study}
}


%%%%%%%%%%%%%%%%%%%%%%%%%%%%%%%%%%%%%%%%%%%%%%%%%%%%%%%%%%%%%%%%%%%%%%%%%%%%%
%%%%

\coverpagestarts


%%%%%%%%%%%%%%%%%%%%%%%%%%%%%%%%%%%%%%%%%%%%%%%%%%%%%%%%%%%%%%%%%%%%%%%%%%%%%
%%%%

\acknowledgements
\noindent

\noindent I would like to thank to RNDr. Marko Genyk-Berezovskyj for his valuable advices.
% TODO: The second person I want to thank to is Ренат Рафисович Кагиров, he make me write this thesis in English during my stay in Красноярск.


%%%%%%%%%%%%%%%%%%%%%%%%%%%%%%%%%%%%%%%%%%%%%%%%%%%%%%%%%%%%%%%%%%%%%%%%%%%%%
%%%%

% TODO: update information
%\declaration{V Kořenovicích nad Bečvárkou dne 15.\,5.\,2008}
\declaration{In Krasnoyarsk on March 27, 2008}


%%%%%%%%%%%%%%%%%%%%%%%%%%%%%%%%%%%%%%%%%%%%%%%%%%%%%%%%%%%%%%%%%%%%%%%%%%%%%
%%%%

\abstractpage

Translation of Czech abstract into English.

\vglue60mm

\noindent{\Huge \textbf{Abstrakt}}
\vspace{8ex}

\noindent
Abstrakt práce by měl velmi stručně vystihovat její podstatu. Tedy čím se práce zabývá a co je jejím výsledkem/přínosem.



%%%%%%%%%%%%%%%%%%%%%%%%%%%%%%%%%%%%%%%%%%%%%%%%%%%%%%%%%%%%%%%%%%%%%%%%%%%%%
%%%%

\tableofcontents
\listoffigures
\listoftables


%%%%%%%%%%%%%%%%%%%%%%%%%%%%%%%%%%%%%%%%%%%%%%%%%%%%%%%%%%%%%%%%%%%%%%%%%%%%%
%%%%

\mainbodystarts

\parskip=5pt plus 4pt minus 4pt
\parindent=0em


%%%%%%%%%%%%%%%%%%%%%%%%%%%%%%%%%%%%%%%%%%%%%%%%%%%%%%%%%%%%%%%%%%%%%%%%%%%%%
%%%%

\chapter{Introduction}

There are algorithms \textit{diffuse} for image processing. Everybody knows, that they work, but nobody knows, how they \textit{exactly} work. The output of this project will be an integrated development environment for writing, debugging and visualizations of these algorithms.

The input for the diffuse algorithm is an image, the output is, for example, in which part of the image is an object situated, where is the background and similar. The algorithm transforms image to a special graph with several vertices for each pixel, the vertices are connected by evaluated edges. Diffuse algorithm works with the local parts of the graph, iteratively updates edges evaluation and is stopped if global system condition is met.

The task of this project can be generalized from manipulation with the images to graphs processing.

The aim of this diploma thesis is to design and implement a special language based on C/C++, which will be used for the graph manipulations. The language has support for basic data types like numbers, floating point numbers, booleans, strings, arrays and structures. Variables can be both local and global, no declarations are needed. Control structures are if-else conditions, for, while and foreach loops, function declarations and calls. Include and define preprocessor directives are implemented at the lexical analyzer layer. The language does not contain support for pointers, object oriented features are not implemented too.

Special data types includes graphs, vertices, edges, vertex sets and edge sets. Interpreter has built-in functions for creating and deleting of these variables, returning neighbors of the vertices and so on. Every vertex and edge can be evaluated by any number of properties of any type, the dot operator \texttt{object.property} can be used for this. Special control structures for processing complex variables are implemented using iterators which are internaly called by foreach loop.

The graphic user interface consists of a text editor with syntax highlighting and a window for the graph visualization during execution of the algorithm. Actual state in the graph window is displayed after the command in the code is executed and graph view can be changed. The graph can be loaded by the code or by the user in the visualization window.

Integrated development environment has also support for debugging which includes breakpoints, code stepping, displaying content of the variables and the call stack.

The command line interpreter is written in C++ language using Bison parser generator (LALR grammar). Graphic user interface is created in Qt library and graph visualizations in OpenGL library.


%%%%%%%%%%%%%%%%%%%%%%%%%%%%%%%%%%%%%%%%%%%%%%%%%%%%%%%%%%%%%%%%%%%%%%%%%%%%%
%%%%

\chapter{Problem description and aim specification}

% TODO: delete
%\begin{itemize}
%\item Popis řešeného problému, vymezení cílů DP/BP a požadavků na implementovaný systém.
%\item Popis struktury DP/BP ve vztahu k vytyčeným cílům.
%\item Rešeršní zpracování existujících implementací, pokud jsou známy.
%\end{itemize}


\section{Application requirements}

Design and implement language interpreter, which will be targeted to the graph manipulations. The language will contain basic data structures necessary for the graphs and will have support for effective writing of the graph algoritms. Interpreter will have graphic user interface consisting of a text editor and a window for graph visualization during execution of the algorithm. Use OpenGL library for the graph visualization.

Graphs without orientation are fundamental, oriented are not needed so much.

\subsection{Language}

The language constructs should be similar to C language. They include basic data types, data and control structures.

\begin{itemize}
\item numbers, floating point numbers, arrays
\item conditions, loops, declarations and functions calling
\item text output to the own console
\item include directive in simple version (not headers but whole files)

\item consider
	\begin{itemize}
	\item pointers and structures like in C language
	\item OOP is not probably needed
	\item simple I/O, mainly reading/writing of the graphs from/to text files
	\item boolean type, maybe not needed
	\item simple strings and characters, no difficult functions for manipulation
	\end{itemize}

\item specific language elements
	\begin{itemize}
	\item primitive types
		\begin{itemize}
		\item graph, vertex, edge, vertex set, edge set
		\end{itemize}
	\item basic functions
		\begin{itemize}
		\item create/delete graph/vertex/edge
		\item evaluate vertex/edge
		\item return all neighbors of the vertex
		\item create/delete vertex/edge set, add/remove vertex/edge
		\item return degree of vertex
		\end{itemize}
	\item control structures
		\begin{itemize}
		\item for all neighbors of vertex execute \texttt{command}
		\item for all vertices/edges in the set execute \texttt{command}
		\item for all verteces/edges in the graph execute \texttt{command}
		\end{itemize}
	\end{itemize}
\end{itemize}


\subsection{Text editor}

The editor should be classic text editor for programmers with syntax highlighting. No other features are reguired now, they will be discussed.


\subsection{Display unit, visualization}

The visualization window display actually processed graph in 3D, it is only graph viewer not editor. User can interactively translate, rotate and resize the graph. Displaying of specific parts of the graph can be enabled and disabled.

If language supports color and/or texture attributes of the vertices/edges/sets, they should be displayed too. Visualization window should have an option for loading of the graph, user can choose to display this graph or graph loaded by the code


\subsection{Graph manipulation, debugger}

Code interpretation can be stepped. Actual state in graph window will be displayed after command is executed and graph view can be changed now. Parameters of the graph during algorithm execution won't be changed, visualization parameters yes (colors, scale, etc.). There is no need to support classic debugging, console for debugging and trace outputs is enough.


\subsection{Minimal system configuration}

\begin{itemize}
	\item CLI version
		\begin{itemize}
		\item i386 and amd64 platform, support for other will be appreciated
		\item GNU/Linux, MS Windows (NT based) or MacOS operating systems, support for other will be appreciated
		\item processor min. 1 GHz
		\item memory min. 2 GB RAM (millions of vertices are expected)
		\end{itemize}

	\item GUI version
		\begin{itemize}
		\item graphic card with OpenGL support (ie. nVidia GeForce 2 and newer)
		\item OpenGL drivers and Qt4 libraries
		\item min. 1024x768 monitor resolution
		\end{itemize}
\end{itemize}


%%%%%%%%%%%%%%%%%%%%%%%%%%%%%%%%%%%%%%%%%%%%%%%%%%%%%%%%%%%%%%%%%%%%%%%%%%%%%
%%%%

\chapter{Analysis and solution suggestions}

% TODO: delete
% Analýza a návrh implementace (včetně diskuse různých alternativ a volby im\-ple\-men\-tač\-ní\-ho prostředí).

\section{General approach}

The general aim of this project is to develop integrated development environment for special programming language that is orientated to the graphs manipulations. This complex task can be divided to the several standalone parts which should be solved in the following order.

\begin{enumerate}
\item Suggest how the graphs will be saved and how will be accessed from the interpreter
\item Design a language for graph manipulations matching the requirements
\item Implement lexical analyzer, parser and command line interpreter
\item Design and implement IDE graphic user interface
\end{enumerate}


\section{Language and libraries for implementation}

There are three commonly used languages for the application implementation - C++, Java and C\#. The third one can be automatically deleted, because of the multiplatform requirement and antipathy of the project author. C++ language is portable at the source code layer and Java programs at the byte code layer so both can be used. No script language like Perl or Python is considered, they are not suitable for such big application.

Great advantages of Java are its memory management and lots of existing libraries. On the other hand most of compilers and iterpreters are written in C or C++ because of compilation to the machine code and the resulting quickness.

The author has a lot of experiences with C++ so it and STL library have been choosen for the implementation of the interpreter code. No other library will be used for base code implementation, command line interpreter will have as less dependencies as possible. It should be portable to all platforms for which C++ compiler exists.

Graphic user interface will be created using Qt library. It is one the best C++ library for GUI programming, multiplatform and free for noncommercial use. The graph visualizations will be done of course in OpenGL which is standard for 3D graphics. Second possible choice, Direct 3D, is not portable to non-Microsoft platforms.


\section{Thoughts about embedded Python}

\begin{verbatim}
Python
	naprogramovat grafové třídy v Pythonu a zkusit použít
	http://www.python.org/doc/ext/embedding.html
	http://docs.python.org/lib/module-pdb.html
	jednoduché na implementaci, ale vlastní interpret jednodušší na použití
\end{verbatim}


\section{Language specifications}

email/01\_zadani\_en.txt

\begin{verbatim}
include("filename");
define("name", "value");


without switch, use if-else instead
without comma operator, i.e. for(a = 0, b = 0; a < 5; a++)
without bit operators
without references

// Function calls
function someFunc(self, parameter)
{

}

someFunc(object, parameter);
object.someFunc(parameter);

Je třeba předávání parametrů odkazem?
	ne, možné uložit data do struktury a vrátit ji

Deklarace proměnných
	jako v C, povinná - hodilo by se při automatickém generování C kódu
	jako v C, nepovinná - tohle asi ne
	nepovinná jako v Javascriptu (var jmeno) - asi zbytečné

Viditelnost proměnných
	neomezovat na úroveň bloků
	viditelnost pouze na úrovni funkce
	globální proměnné pomocí buildin funkcí

Data types (internal)
	no declarations
	no "const" modifier
	null
	bool
	int
	float
	string
	array, struct
		graph
		vertex, vertex iterator
		edge, edge iterator

Conditions, loops, etc. (c-style)
	if
	if-else
	for
	while
	switch
	foreach(value; iterableObject) doSomething(value);

Structured jumps (c-style)
	return
	break
	continue

Buildin functions
	isDefined()
	isVARIABLETYPE() query
	toVARIABLETYPE() casts
	setGlobal('variable name', variable)
	getGlobal('variable name')
	echo()/print()
	exit()
	rand()
	math, string, file, etc.
\end{verbatim}

Rozhodl jsem se, ze switch nakonec nebudu implementovat. Standardni z C++ je kvuli breakum relativne slozity na vytvoreni, a proto jsem uvazoval o nejakem zjed\-no\-du\-se\-nem ve stylu

\begin{verbatim}
switch(E)
{
	case(num|str)
		cmd();

	case(num|str)
	{
		cmd();
		cmd();
	}

	default
		cmd();
}
\end{verbatim}

Potom jsem si rekl, ze by se case mohlo prejmenovat na if a default na else (aby bylo mene klicovych slov), napsal si to a dosel k nazoru, ze to na prvni pohled vypada uplne stejne jako nekolik obycejnych if-else za sebou akorat navic zanorene v sekci swich :-)

\subsection{Basic idea}

\subsection{Standard constructs}


\subsection{Data containers}

\begin{verbatim}
// Array, stack, queue, map

a = new array();

a[0] = 10;
a['name'] = 'value';
a.lenght();

a.pushBack(item);
a.getBack();
a.popBack();

a.pushFront(item);
a.getFront();
a.popFront();

a.push(position, item);
a.get(position);
a.pop(position);

\subsection{Special constructs, graphs support}

TODO: X36ASS

Množina vrcholů, množina hran
	TODO: jsou na to použitelné iterátory + vlastnosti?

//---------------- Color all vertex neighbors to red
vi = v.getNeighbors();
while(vi.hasNext())
{
	tmp = vi.getNext();
	tmp.setProperty('color', 'red');
}

// ...or using foreach
foreach(neighbor; v.getNeighbors())
	neighbor.setProperty('color', 'red');



//---------------- Recursive depth first search
function dfs(v)
{
	v.setProperty('visited', true);

	foreach(i; v.getNeighbors())
	{
		DOBEFORE(i);
		if(i.getProperty('visited') == false)
			dfs(i);
		DOAFTER(i);
	}
}
\end{verbatim}


\subsection{Code examples}

Ukázka existujících algoritmů (zkusit si je přepsat do navrženého jazyka)

\section{Graph representation analysis}

The assumed size of the stored graphs is millions vertices and edges. Each graph primitive will be a variable in the interpreter and may be referred from the several places so same kind of memory management should be propounded. The work with the whole graph and with its parts represented by sets should be similar.


\begin{verbatim}
Uvažovat paralelní hrany???!!!
	trojrozměrné pole hran u matice sousednosti
	jak označovat hrany? [počátek][konec] už není jednoznačné
	TODO: nahradit matici sousednosti za nějaký druh spojové
		reprezentace? Množiny hran apod. by se takhle
		implementovaly asi líp...

Reprezentace grafu
	maticová reprezentace není možná kvůli paralelním hranám,
		počáteční a koncový vrchol by neurčoval hranu
		jednoznačně, další problémy, preferovaná univerzálnost
	pole vrcholů a pole hran
	jednoduché na implementaci
	jednoduché vracení množin vrcholů a hranám
	pomalé? nepředpokládají se extrémně velké grafy
\end{verbatim}


\subsection{Standard representations}

\subsection{Own representation}

\subsubsection{Memory management}

\subsection{Summary}


\section{Variables in the script, types and operations}

\begin{verbatim}
proměnné
	hierarchie tříd, ne jen jedna třída
	double dispatch pattern - http://en.wikipedia.org/wiki/Double\_dispatch
	Bruce Eckel: Thinking in C++ 2, Multiple dispatching, str. 459
\end{verbatim}

\subsection{Class hierarchy}

\subsection{Multiple dispatching pattern}


\section{Script representation and execution}


\subsection{AST traversion}

\begin{verbatim}
Jak uložit pozici ve zdrojových kódech?
	zaznamenat si pozici příkazu, bloku, funkce a ukládat to do
	těchto tříd ve stromě lexan vrací změnu pozice jako token,
	virtuální instrukce, problémy s gramatikou?
\end{verbatim}


\subsection{Virtual machines}

\subsection{Script debugging support}

\begin{verbatim}
Implementace krokování
	rekurzivní volání execute() na stromovou strukturu, krokování pomocí
	synchronizace vláken lineární pole instrukcí, virtuální procesor

breakpoint(bool);
\end{verbatim}


\section{Memory leaks, memory manager}

\subsection{Garbage collector}

\subsection{Smart pointers}

\begin{verbatim}
auto\_ptr ze STL nelze použít, protože...
http://www.davethehat.com/articles/smartp.htm
http://ootips.org/yonat/4dev/smart-pointers.html
http://www.josuttis.com/libbook/cont/countptr.hpp.html
\end{verbatim}

\section{Graphic user interface}


\section{Code generation}

\begin{verbatim}
Automatické generování C/Java zdrojových kódů?
	aneb po zkušenostech s GWT proč ne? :-)
	komentáře asi ignorovat
\end{verbatim}


\section{Summary}



%%%%%%%%%%%%%%%%%%%%%%%%%%%%%%%%%%%%%%%%%%%%%%%%%%%%%%%%%%%%%%%%%%%%%%%%%%%%%
%%%%

\chapter{Realization}

% TODO: delete
Popis implementace / realizace se zaměřením na nestandardní části řešení.

\section{Base code and command line interpreter}

\subsection{Static object initialization and memory leaks}

\begin{verbatim}
init\_static\_objects.cpp
BaseObject
general.hpp ???
\end{verbatim}


\subsection{BaseObject::dump()}

\subsection{Lexical analyzer}

Z větší části převzatý Borsch a rozšířený
http://woq.nipax.cz/borsch/devel.php

\begin{verbatim}
// Identifikatory, retezec -> id
// Slozitost: generovani, pouziti id, zpetne dotazovani na retezce
http://zdrojak.root.cz/clanky/javascript-implementace-v-prohlizecich/
http://links.twibright.com/doc/projektova.pdf - links projektová
	dokumentace
http://links.twibright.com/doc/vyvojova.pdf - vývojová dokumentace
\end{verbatim}


\subsection{Grammar, Bison}

\subsubsection{Positions in the script and AST}


\subsection{Script variables, Value* class hiearchy}

\subsection{Abstract syntax tree, Node* class hierarchy}


\subsection{Functions}

\subsubsection{Functions written in the script}

\subsubsection{Built-in functions}


\subsection{Code generators}

\subsubsection{Operators, built-in functions}

\subsubsection{C/C++ macros}
Great problems with debugging

\subsubsection{Source code generators}


\subsection{Script debugging}

\subsubsection{Code execution}

Recursive traversion of AST, mutexes


\subsubsection{Breakpoint() built-in function}

\subsubsection{Script variables}

\subsubsection{Call stack}


\subsection{Summary}



\section{Graphic user interface}

\subsection{Main application}

\subsection{Editor}

Code Editor Example
http://doc.trolltech.com/4.5/widgets-codeeditor.html

\subsection{Debugging}

\subsection{Graph visualization}

\subsection{Summary}



%%%%%%%%%%%%%%%%%%%%%%%%%%%%%%%%%%%%%%%%%%%%%%%%%%%%%%%%%%%%%%%%%%%%%%%%%%%%%
%%%%

\chapter{Testing}

% TODO: delete
\begin{itemize}
 \item Způsob, průběh a výsledky testování.
 \item Srovnání s existujícími řešeními, pokud jsou známy.
\end{itemize}

\section{Unit tests}

\section{Script unit tests}

\section{Benchmarks}

\section{Usability}


%%%%%%%%%%%%%%%%%%%%%%%%%%%%%%%%%%%%%%%%%%%%%%%%%%%%%%%%%%%%%%%%%%%%%%%%%%%%%
%%%%

\chapter{Conclusion}

% TODO: delete
\begin{itemize}
\item Zhodnocení splnění cílů DP/BP a  vlastního přínosu práce (při formulaci je třeba vzít v potaz zadání práce).
\item Diskuse dalšího možného pokračování práce.
\end{itemize}


%%%%%%%%%%%%%%%%%%%%%%%%%%%%%%%%%%%%%%%%%%%%%%%%%%%%%%%%%%%%%%%%%%%%%%%%%%%%%
%%%%

%\bibliographystyle{abbrv}
%bibliographystyle{plain}
%\bibliographystyle{psc}
%\bibliographystyle{alpha} % [AutorRok]
%{
%\def\CS{$\cal C\kern-0.1667em\lower.5ex\hbox{$\cal S$}\kern-0.075em $}
%\bibliography{reference}
%}


%%%%%%%%%%%%%%%%%%%%%%%%%%%%%%%%%%%%%%%%%%%%%%%%%%%%%%%%%%%%%%%%%%%%%%%%%%%%%
%%%%

\appendix


\chapter{Language gammar}


%%%%%%%%%%%%%%%%%%%%%%%%%%%%%%%%%%%%%%%%%%%%%%%%%%%%%%%%%%%%%%%%%%%%%%%%%%%%%
%%%%

\chapter{Class hierarchy}


%%%%%%%%%%%%%%%%%%%%%%%%%%%%%%%%%%%%%%%%%%%%%%%%%%%%%%%%%%%%%%%%%%%%%%%%%%%%%
%%%%

\chapter{Build environment}

\section{Development tools, versions}
\section{Compiling}
\section{Execution, command line parameters}
\section{Graphic user interface}

base code for the application qt4/examples/mainwindows/mdi


%%%%%%%%%%%%%%%%%%%%%%%%%%%%%%%%%%%%%%%%%%%%%%%%%%%%%%%%%%%%%%%%%%%%%%%%%%%%%
%%%%

\chapter{List of abbreviations}
\begin{description}
\setlength{\labelwidth}{2.5cm}
\setlength{\itemindent}{1.5cm}
\item[LALR] Look Ahead Left to Right (grammar type)
\item[AST] Abstract Syntax Tree
\item[CLI] Command Line Interface
\item[GUI] Graphics User Interface
\item[OOP] Object Oriented Programming
\item[I/O] Input/Output
\item[OpenGL] Open Graphics Library
\end{description}
\vdots


%%%%%%%%%%%%%%%%%%%%%%%%%%%%%%%%%%%%%%%%%%%%%%%%%%%%%%%%%%%%%%%%%%%%%%%%%%%%%
%%%%

\chapter{UML diagrams}

TODO: X36ASS


%%%%%%%%%%%%%%%%%%%%%%%%%%%%%%%%%%%%%%%%%%%%%%%%%%%%%%%%%%%%%%%%%%%%%%%%%%%%%
%%%%

\chapter{Install and user guide}


%%%%%%%%%%%%%%%%%%%%%%%%%%%%%%%%%%%%%%%%%%%%%%%%%%%%%%%%%%%%%%%%%%%%%%%%%%%%%
%%%%

\chapter{Content of the attached CD}


\end{document}
